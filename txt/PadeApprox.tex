\documentclass[a4paper,14pt]{article}
\usepackage{tikz}
\usepackage[utf8]{inputenc}
\usepackage[russian]{babel}
\usepackage{amsmath}
\usepackage{listings}
\usetikzlibrary{calc, matrix}

%{{{Command defns
\newcommand{\definition}[2]
{
	\textbf{#1} -- #2\\
}
\newlength{\drop}
\newcommand*{\titleMS}{\begingroup
	\drop=0.1\textheight
	\vspace*{\drop}
	\centering
	{\LARGE THE UNIVERSITY}\\[2\baselineskip]
	{\LARGE\sffamily Conundrums:\\puzzles for the mind}\par
	\vfill
	{\LARGE Построение аппроксимирующей дробно-рациональной функции методом Паде}\par
	\vspace{\drop}
	{\large some remarks}\par
	\vfill
	{\large\bfseries The candidate}\par
	\vspace*{\drop}
\endgroup}
%}}} 

\begin{document}
\titleMS
\clearpage
\section{Понятия и определения} % (fold)
\definition{Рациональная функция}{дробь, числителем и знаменателем которой являются многочлены}
\definition{Аппроксимация (приближение)}{научный метод, состоящий в замане одних объектов другими, 
в том или ином смысле близкими к исходным, но более простыми}
\section{Постановка задачи} % (fold)
Требуется написать программу, строящую по коэффициентам разложения 
исходной функции в ряд Маклорена её приближение рациональной функцией
методом Паде и выводящую результат в формате $\TeX$.Коэффициенты разделены во входном файле пробелами.
\section{Описание метода} % (fold)
Исходная функция $f(x)$ представляется своим рядом Тейлора:\\
\[
	f(x) = c_0 + c_1x + c_2x^2 + ... + c_kx^k
\]
Приближающая функция с многочленом степени $L$ в числителе и $M$ в знаменателе обозначается 
$R_{[L/M]}(x)$ и выглядит следующим образом: 
\[
	R_{[L/M]}(x) = 
	\frac{a_0 + a_1x + a_2x^2 + ... + a_Lx^L}
		{b_0 + b_1x + b_2x^2 + ... + b_Mx^M}
\]

$b_0$ в знаменателе можно заменить на 1 не нарушая общности, тем самым сократив количество коэффициентов:
\[
	R_{[L/M]}(x) = 
	\frac{a_0 + a_1x + a_2x^2 + ... + a_Lx^L}
		{1 + b_1x + b_2x^2 + ... + b_Mx^M}
	\]
	Далее:
	\begin{gather*}
			f(x) = R(x) \nonumber \\
			\frac{a_0 + a_1x + a_2x^2 + ... + a_Lx^L}
			{1 + b_1x + b_2x^2 + ... + b_Mx^M} = 
			c_0 + c_1x + c_2x^2 + ... + c_kx^k \nonumber \\
			a_0 + a_1x + a_2x^2 + ... + a_Lx^L = 
			(c_0 + c_1x + c_2x^2 + ... + c_kx^k)
			(1 + b_1x + b_2x^2 + ... + b_Mx^M) \nonumber
	\end{gather*}

последнее равенство эквивалентно системе уравнений:\\
\[
	\left\{
		\begin{array}{l}
			a_0 = c_0\\ 
			a_1 = c_0b_1 + c_1\\
			a_2 = c_0b_2 + c_1b_1 + c_2\\
			\ldots\\
			a_L = c_0b_L + c_1b_{L-1} + ... + c_L\\
			0 = b_Mc_{L-M+1} + b_{M-1}c_{L-M+2} + ... + b_1c_L + c_{L+1}\\
			\ldots\\
			0 = b_Mc_L + b_{M-1}c_{L+1} + ... + b_1c_{L+M-1} + c_{L+M}
		\end{array}
		\right. \\
	\]
Здесь $b_i, i>M$ может быть неопределен, если $L>M$. В этом случае просто 
положим такие $b_i$ равными нулю.

Перепишем систему так, чтобы неизвестные оказались слева:
\[
	\left\{
		\begin{array}{l}
			a_0 = c_0\\ 
			a_1 - c_0b_1 = c_1\\
			a_2 - c_1b_1 - c_0b_2 = c_2\\
			\ldots\\
			a_L - b_1c_{L-1} - b_2c_{L-2} - ... - b_Lc_0=  c_L\\
			- b_1c_L - b_2c_{L-1} - ... - b_Lc_1 = c_{L+1}\\
			\ldots\\
			- b_1c_{L+M-1} - b_2c_{L+M-1} - ... - b_Mc_{L} = c_{L+M} 
		\end{array}
		\right. \\
	\]

Системам соответствует матрица:
$$
	\begin{pmatrix}
		1	&	0	&	0	&	0	&	\ldots	&	0	&	0	&	0	&	\ldots	&	0	\\
		0	&	1	&	0	&	0	&	\ldots	&	0	&	-c_0	&	0	&	\ldots	&	0	\\
		0	&	0	&	1	&	0	&	\ldots	&	0	&	-c_1	&	-c_0	&	\ldots	&	0	\\
		\ldots	&	\ldots	&	\ldots	&	\ldots	&	\ldots	&	\ldots	&	\ldots	&	\ldots	&	\ldots	&	\ldots	\\
		0	&	0	&	0	&	0	&	\ldots	&	1	&-c_{L-1}	&-c_{L-2}	&	\ldots	&-c_{L-M}	\\
		0	&	0	&	0	&	0	&	\ldots	&	0	&	-c_L	&-c_{L-1}	&	\ldots	&-c_{L-M+1}	\\
		\ldots	&	\ldots	&	\ldots	&	\ldots	&	\ldots	&	\ldots	&	\ldots	&	\ldots	&	\ldots	&	\ldots	\\
		0	&	0	&	0	&	0	&	\ldots	&	0	&-c_{L+M-1}	&-c_{L+M-2}	&	\ldots	&-c_{L}	\\
	\end{pmatrix}
	$$
Решив систему уравнений методом Гаусса, найдем коэффициенты многочленов.
\clearpage
\section{Программа}
\textbf{main.cpp}
\lstinputlisting[language = C++, firstline = 10]{../main.cpp}
\clearpage

\textbf{calc.cpp}
\lstinputlisting[language = C++, firstline = 7]{../calc.cpp}
\clearpage

\textbf{gaussian\_el.cpp}
\lstinputlisting[language = C++, firstline = 6, lastline=78]{../gaussian_el.cpp}
\clearpage

\textbf{texport.cpp}
\lstinputlisting[language = C++, firstline = 6]{../texport.cpp}
\clearpage

\end{document}
