\documentclass[a4paper,14pt]{article}
\usepackage[utf8]{inputenc}
\usepackage[russian]{babel}

%{{{Command defns
\newcommand{\definition}[2]
{
	\textbf{#1} -- #2\\
}
%}}} 
\begin{document}
\section*{Построение аппроксимирующей дробно-рациональной функции методом Паде}
\subsection{Понятия и определения} % (fold)
\definition{Дробно-рациональная функция}{what it is}
\definition{Аппроксимация}{what it is}
\subsection{Постановка задачи} % (fold)
Требуется написать программу, строющую по коэффициентам разложения 
исходной функции в ряд Тейлора её приближение дробно-рациональнной функцией
методом Паде. Коэффициенты разделены во входном файле пробелами.
\subsection{Описание метода} % (fold)
Исходная функция $f(x)$ представляется своим рядом Тейлора:\\
\[
	f(x) = c_0 + c_1x + c_2x^2 + ... + c_kx^k
\]
Приближающая функция с многочленом степени L в числителе и M в знаменателе обозначается 
$R_{[L/M]}(x)$ и выглядит следующим образом: 
\[
	R_{[L/M]}(x) = 
	\frac{a_0 + a_1x + a_2x^2 + ... + a_Lx^L}
		{b_0 + b_1x + b_2x^2 + ... + b_Mx^M}
\]

$b_0$ в знаменателе можно заменить на 1 не нарушая общности, тем самым сократив количество коэффициентов:
\[
	R_{[L/M]}(x) = 
	\frac{a_0 + a_1x + a_2x^2 + ... + a_Lx^L}
		{1 + b_1x + b_2x^2 + ... + b_Mx^M}
\]
Далее:
\[
	\begin{array}{l}
		f(x) = R(x)\\
		\frac{a_0 + a_1x + a_2x^2 + ... + a_Lx^L}
		{1 + b_1x + b_2x^2 + ... + b_Mx^M} = 
		c_0 + c_1x + c_2x^2 + ... + c_kx^k\\
		a_0 + a_1x + a_2x^2 + ... + a_Lx^L = 
		(c_0 + c_1x + c_2x^2 + ... + c_kx^k)
		(1 + b_1x + b_2x^2 + ... + b_Mx^M)
	\end{array}
\]

последнее равенство эквивалентно системе уравнений:\\
\[
	\left\{
		\begin{array}{l}
			a_0 = c_0\\ 
			a_1 = c_0b_1 + c_1\\
			a_2 = c_0b_2 + c_1b_1 + c_2\\
			\ldots\\
			a_L = c_0b_L + c_1b_{L-1} + ... + c_L\\
			0 = b_Mc_{L-M+1} + b_{M-1}c_{L-M+2} + ... + b_1c_L + c_{L+1}\\
			\ldots\\
			0 = b_Mc_L + b_{M-1}c_{L+1} + ... + b_1c_{L+M-1} + c_{L+M}
		\end{array}
		\right. \\
	\]
Здесь $b_i, i>M$ может быть неопределен, если $L>M$. В этом случае просто 
положим такие $b_i$ равными нулю.

Перепишем систему так, чтобы неизвестные оказались слева:
\[
	\left\{
		\begin{array}{l}
			a_0 = c_0\\ 
			a_1 - c_0b_1 = c_1\\
			a_2 - c_1b_1 - c_0b_2 = c_2\\
			\ldots\\
			a_L - b_1c_{L-1} - b_2c_{L-2} - ... - b_Lc_0=  c_L\\
			- b_1c_L - b_2c_{L-1} - ... - b_Lc_1 = c_{L+1}\\
			\ldots\\
			- b_1c_{L+M-1} - b_2c_{L+M-1} - ... - b_Mc_{L} = c_{L+M} 
		\end{array}
		\right. \\
	\]

Системам соответствует матрица:
\[
\begin{matrix}
	1&2&3&4&5&6\\1$2$3$4$5$6
\end{matrix}
\]
\end{document}
